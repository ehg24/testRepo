\documentclass{ximera}

\usepackage{tikz}
\usepackage{tkz-euclide}
\usetkzobj{all}


\outcome{Understand a first example of the Ximera style.}
\outcome{Have a nice basic example to work from.}

\title{First example}

\begin{document}
\begin{abstract}
In this activity we see some examples.
\end{abstract}
\maketitle

To start we can have theorem environments:

\begin{theorem}
Given a right triangle:
\begin{image}
\begin{tikzpicture}
\coordinate (A) at (0,2);
\coordinate (B) at (0,5);
\coordinate (C) at (6.5,2);
\tkzMarkRightAngle(C,A,B)
\tkzDefMidPoint(A,B) \tkzGetPoint{a}
\tkzDefMidPoint(A,C) \tkzGetPoint{b}
\tkzDefMidPoint(B,C) \tkzGetPoint{c}
\draw (A)--(B)--(C)--cycle;
\tkzLabelPoints[above](c)
\tkzLabelPoints[below](b)
\tkzLabelPoints[left](a)
\end{tikzpicture}
\end{image}
We have that:
\[
a^2 + b^2 = c^2
\]
\end{theorem}

\begin{exercise}
$3\times 2=\answer{6}$
\end{exercise}



\end{document}
